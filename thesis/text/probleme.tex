\newpage
\section{Probleme}
Dieser Abschnitt widmet sich den Problemen, die erst bei der Durchführung dieser Arbeit aufgetreten sind und vorher nicht abzuschätzen waren.

\subsection{Kommunikation Master -- Slave}
Die Idee der drahtlosen Kommunikation im Rahmen dieser Bachelorarbeit bestand darin, dass Steuerkommandos und Daten immer getrennt gesendet und immer auf 32-Bit aligned werden. Das hat zur Folge, dass Daten immer vom vorherigen Steuercode abhängig sind. Kommt der Steuercode nicht beim Empfänger an, können nachfolgende Daten dem nicht zugeordnet bzw. korrekt interpretiert werden. Die Kommunikation war serialisert -- die nächsten Daten konnten nur interpretiert werden, wenn die vorherigen Daten fehlerfrei ankamen. Da es häufig zu Verbindungsabbrüchen während der Kommunikation kam, wurde diese Kommunikationsvariante durch Structs abgelöst. Dabei enthält das Struct den Steuercode und die dazugehörigen Daten. Somit sind Steuercodes und Daten zusammen. Gehen nun Daten verloren, können nachfolgende Daten weitherhin interpretiert werden, denn der dazugehörige Steuercode ist vorhanden/ bekannt. Die Kombination von Daten und Steuercodes ermöglicht eine variable Kommunikation. Damit ist man nicht mehr abhängig von den vorherigen Daten.

\subsection{Genauigkeit -- Zeitsynchronisation}
Im Rahmen der Zeitsynchronisation kam es immer wieder zu dem Problem, dass die Genauigkeit bei mehrfachen Wiederholungen nicht besser wurde. Somit ist es unerheblich, ob eine Zeitsynchronisation einmal oder mehrmals stattfindet. Dadurch, dass sich die Zeitsynchronisation nicht im \si{\mu}-Sekundenbereich auflöst, kann auch keine Messung ohne große Abweichung erfolgen. Die Abweichung kommt insbesondere durch die Aufrufe $getSystemTime()$ und $udp\_send\_packet()$ zustande. Die Systemzeit wird vor dem Aussenden vom Programm selber in das Datenpaket eingefügt und nicht von der Firmware des Funksenders. Weitherhin baut die Funktion $udp\_send\_packet()$ zuerst das UDP-Paket zusammen, welches wiederrum Zeit kostet, bis es gesendet werden kann. Eine weitere Vermutung für die Ungeauigkeit ist, dass aufgrund von Störsignalen keine \si{\mu}-Sekundenauflösung für die Zeitsynchronisation erreicht werden kann.

\subsection{Messgenauigkeit}
Theoretisch ist nach einer Zeitsynchronisation bekannt, um wieviel Zeit der Master dem Slave hinterherhängt, bzw. vorraus ist. Bei den Distanzmessungen mit dem \board \platz kam es dabei immer wieder zu großen Distanzschwankungen. Diese haben drei Ursachen: eine nicht im \si{\mu}-Sekundenbereich auflösende Zeitsynchronisation, eine Verzögerung der Messinstrumente sowie eine Ablenkung des Schalls vom Aussenden bis zum Empfangen des Mikrofons.
