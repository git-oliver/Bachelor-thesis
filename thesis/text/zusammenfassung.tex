\newpage
\section{Zusammenfassung}

Das Ziel dieser Arbeit, eine Positionsbestimmung im Zentimeterbereich durchzuführen, konnte nicht vollständig erreicht werden. Ein Grund war die mir nur begrenzt zur Verfügung stehende Zeit sowie die Einschränkung bei der Leistungsfähigkeit der Hardware.
Die Arbeit hat verschiedene Aspekte der Positionsbestimmung untersucht. Das Betriebssystem RIOT, die Hardware \microphone \platz und den \ultraschall. Dabei wurde die Verzögerung von Eingangs- und Ausgangssignals untersucht.
\\
Die Ergebnisse haben ergeben, dass RIOT sich aufgrund der starken Verzögerung des Funktionsaufrufs $getSystemTime()$ nicht eignet für eine Positionsbestimmung. Weiterhin wurde festgestellt, dass der Ultraschallsensor nicht geeignet, da er direkten Sichtkontakt benötigt und nur einen Sendekegel von \SI{15}{\degreeCelsius} hat. Eine Vermutung für die nicht vollständige Positionsbestimmung ist, dass der Schall zwischen dem Raum des Sound Detectors und dem Lautsprecher abgelenkt wird und nicht auf den direkten Weg zum Mikrofon transportiert wird. Für die Positionsbestimmung müssen Master und Slave eine synchrone Zeit haben. Um dies zu gewährleisten, wurde mit PTP eine Zeitsynchronisation implementiert. Dabei konnte nachgewiesen werden, dass ohne Hardwareunterstützung keine genaue Zeitsynchronisation möglich ist, die für eine Positionsbestimmung ausreichend ist. Ohne eine Hardwareunterstützung kann eine Genauigkeit von wenigen Millisekunden erreicht werden. Dies spiegelt auch mein Ergebnis wider. Für zukünftige anwendungsorientierte Forschungen wäre es hilfreich, die beschriebenen Module zu wiederholen, um eventuelle neue Abweichungen diagnostizieren zu können. Darüber hinaus zeigt dieses Projekt, das die Umsetzung der Theorie in die Praxis für die Zukunft eine Herausforderung darstellt. Die Software liegt der Arbeit bei, sowie auf Github \cite{src_GITHUB_CODE_BA}\cite{src_genauigkeit_zeit_sync} .

