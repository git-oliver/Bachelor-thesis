\newpage
\section{Einleitung}

Diese Arbeit hat das Ziel, eine Positionsbestimmung auf Basis von Schalllaufzeitmessungen zu entwickeln und aufzubauen. Sie soll die Grundlage sein für eine fehlerfreie Fahrt in Gebäuden mit dem hausinternen CE-Car sein \cite{src_CE_CAR}. Die Positionsgenauigkeit spielt dabei eine untergeordnerte Rolle. Positionierungssysteme gibt es viele; allerdings sind diese nicht immer frei verfügbar, kostenintensiv und oft nur für den Outdoorbereich entwickelt worden \cite{src_INDOOR_OUTDOOR}. Ziel ist es, ein mobiles eingebettetes System für den Indoorbereich zu entwicklen, mit dem Fokus auf geringe Kosten. Damit können Modellautos, Drohnen oder Robotor ausgestattet werden. Darüber hinaus muss das System ohne externe Dienste oder Netzanbindung funktionstüchtig sein. Zudem muss das System einfach erweiterbar sein, um der zukunftigen Entwicklung Schritt zu halten. Die Validierung erfolgt durch eine prototypische Anwendung.

\subsection{Aufbau der Arbeit}
Die Bachelorarbeit ist folgendermaßen gegliedert: Zuerst werden in Kapitel 2 die theoretischen Grundlagen erläutert, sowie die verwendete Hardware und Software beschrieben. Darauf aufbauend beschreibt Kapitel 3, die Implemtierung. Im Anschlus wird das System evaluiert. Zum Schluss werden Probleme aufgezeigt, sowie ein Ausblick für mögliche Erweiterungen des Systems gegeben.
