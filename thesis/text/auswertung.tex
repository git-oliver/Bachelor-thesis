\newpage
\section{Auswertung}

Nachdem die Implementierung abgeschlossen ist, widmen wir uns in diesem Kaptiel der Auswertung der Daten.

\subsection{Hardware}
Die aufgetretenen Abweichungen entstanden meistens, wenn die Peripherie Daten generiert hatte bzw. angesprochen wurde. Da viele dieser Abweichungen konstant sind, kann ein Offset für ein korrektes Ergebnis mit in das Ergebnis einfließen. Allerdings sind einige Abweichungen nicht erklärbar, was die Bestimmung eines Offset erschwert. Weiterhin zeigt sich, dass die verwendete Hardware nicht optimal auf dieses Problem abgestimmt ist. Dazu zählt der Tongeber und der \microphone . RIOT sieht sich zwar als Echtzeitbetriebssystem, allerdings sind zu viele Schichten zwischen dem laufenden Programm und der Hardware vorhanden, da ein Systemaufruf für die Abfrage der Systemzeit nicht über \SI{7,21}{\mu s} dauern. Dies verfälscht die Systemzeit, die eigentlich \si{\mu}-Sekunden genau sein soll. Des weiteren zeigt sich, je weniger Komponenten (Peripherie) verwendet werden, umso einfacher ist die Fehlersuche.

\subsection{Messergebnis}
Die Messergebnisse zeigen, dass eine Positionsbestimmung wie sie durchgeführt wurde nicht genau genug ist. Die Abweichung, die das Messergebnis hauptsächlich verfälscht, liegt beim Tongeber und dem Mikrofon. Weil kein Muster bei der Abweichung erkennbar ist, kann kein Offset bestimmt werden. Die Verwendung von fehlerbehafteten Bausteinen kann ausgeschlossen werden, da das Oszilloskop für alle verwendeteten Mikrofone die gleichen Abweichungen aufzeichnet. Allerdings ist RIOT bei dem Aufruf der Interruptroutinen nahezu konstant. 

\subsection{Software}
Die Software kann eine Positionsbestimmung durchführen; dabei ist sie allerdings auf korrekte Distanzmesswerte angewiesen. Da diese fehlen, kann die Software nur durch Unit-Tests validiert werden. Weiterhin zeigt sich, da die Software in Module aufgeteilt wurde, dass die einzelnen Module perfekt zusammen agieren, auch wenn die Distanzmesswerte fehlerbehaftet sind.

\subsection{RIOT}
Im Verlauf dieser Arbeit zeigt sich, dass RIOT eher hinderlich als hilfreich ist. Da zwischen der Hardware und dem ausführbaren Code ein Betreibssystem eingeführt wurde, kommt es zwingend zu weiteren Verzögerungen. Weiterhin bietet RIOT nicht die Möglichkeit einer Interruptroutine beim Eintrefen von drahtlosen Nachrichten. Des weiteren bietet RIOT das Mitsenden der Systemzeit bei drahtlosen Nachrichten nicht an. Somit wird wertvolle Zeit verschwendet. Das \board \platz wird von RIOT nur rudimentär unterstüzt. Somit können nicht alle Funktionen des \board \platz verwendet werden.

\subsection{Zeitsynchronisation}
Die Abweichungen bei der Zeitsynchronisation kommen dadurch zu Stande, dass die Zeitsynchronisation in Software implementiert wurde und keine Hardwarebeschleuniger verwendet wurden. Somit kommt es immer zu einer Verzögerung von meheren Millisekunden. Mithilfe eines Hardwarebeschleuniger können auch diese Verzögerungen eliminiert werden. Weiterhin zeigt sich, dass diese Arbeit alle Anforderungen erfüllt für den Einsatz des PTP.

\subsection{Mathematischer Hintergrund}
Die Gleichungen die für eine Positionsbestimmung zwei dimensionaler Raum verwendet werden, können einfach für \si{N}-Dimensionen erweitert werden. Pro Dimension kommt eine Gleichung hinzu und diese müssen gleichgesetzt werden. 















